\Mycomment{Pr�senter un �tat de l'art national et international, en dressant l'�tat des connaissances sur le sujet. 
Faire appara�tre d'�ventuelles contributions des partenaires de la proposition de projet � cet �tat de l'art.
Faire appara�tre d'�ventuels r�sultats pr�liminaires. 
Inclure les r�f�rences bibliographiques n�cessaires en annexe (cf. \masec{refs}).
}

\subsubsection{Mod�lisation/outils pour le Mat�riel et logiciel et choix SysML}
\redac{IRIT, LIFC}
\vspace{1cm}
\subsubsection{V�rification des mod�les SysML}
\redac{IRIT}
\vspace{1cm}
\subsubsection{Ing�nierie des exigences et leurs transformations}
\redac{???}
\vspace{1cm}
\subsubsection{Transformation de mod�le vers mod�le mat�riel}
\redac{LISI, LIFC}
\vspace{1cm}
\subsubsection{Mod�lisation/outils pour expliquer le choix de VHDL-AMS et/ou Verilog et/ou SystemC et/ou ROSETTA}
\redac{INESS, FEMTO-ST}
\vspace{1cm}
\subsubsection{V�rification de propri�t�s mat�rielles choix PSL et/ou autres}
\redac{IRIT, LIFC, INESS}
\vspace{1cm}
\subsubsection{G�n�ration de Tests}
\redac{LIFC}
\vspace{1cm}
