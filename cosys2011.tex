%====================================
% Titre                 : Exemple de rapport de projet
% Last Update           : 25/01/2011
% 						: first draft taken from the CPP project (http://www.lix.polytechnique.fr/~bouissou/cpp/index.php?n=Main.Documents)
%====================================
\documentclass[oneside]{article} % or report
%====================================
\usepackage{a4wide}
\usepackage[francais,public]{anr}
% options:
% lang: francais ou english (defaut: francais)
% classification: public, restricted, confidential (defaut: public).
\usepackage[french]{babel}
\usepackage[latin1]{inputenc}
\usepackage{latexsym,times}
\usepackage{tabularx}
\usepackage[sc]{mathpazo}
\linespread{1.05}         % Palatino needs more leading (space between lines)

\newcommand\Mycomment[1]%          %  temporary remark for the
 {\par\smallskip                     %  authors:
  \begin{center}%                    %
   \fbox%                            %    --------
   {\parbox{0.9\linewidth}%          %    |  #1  |
    {\raggedright--- 
	\textcolor{red}{Note :} #1}%     %    --------
   }%                                %
  \end{center}%                      %
  \par\smallskip                     %
 }
%\renewcommand\Mycomment[1]{}
\date{\today}
\author{Jean-Michel Bruel}
\project{MACAO, IRIT, CNRS}
\reference{0.0}
\title{Un exemple de rapport de projet}
\version{1}
\status{Draft}

%--------------------------- Notes (si besoin) ---------------------------
%\begin{notes} % used with \showNotes
%Ceci est une note
%\end{notes}
%-------------------------------------------------------------------------

%===========================================================
\begin{document}
\pagestyle{fancy}
%\showNotes
\makecosys
~\\\vspace{1cm}
\tableofcontents
%===========================================================
\Mycomment{les encadr\'es comme celui-ci sont produits avec une
  commande \verb:comment:, son but est d'inclure des commentaires pour les
  co-auteurs du document.  Ils doivent dispara\^{\i}tre dans la version finale
  du document.}

%===========================================================
\masection{R\'esum\'e de la proposition}{Executive summary}\label{sec:intro}
%===========================================================
\Mycomment{Recopier le r�sum� utilis� dans le document administratif et financier (dit document de soumission)}

%===========================================================

%===========================================================
\masection{Contexte, positionnement et objectifs de la proposition}{Context, position and objectives of the proposal}\label{sec:contexte}
%===========================================================
\Mycomment{A titre indicatif : de 5 � 10 pages pour ce chapitre.

Pr�sentation g�n�rale du probl�me qu'il est propos� de traiter dans le projet et du cadre de travail (recherche fondamentale, industrielle ou d�veloppement exp�rimental).}

\masubsection{Contexte et enjeux �conomiques et soci�taux}{Context, social and economic issues}\label{sec:enjeuxEco}
\Mycomment{D�crire le contexte �conomique, social, r�glementaire \ldots dans lequel se situe le projet en pr�sentant une analyse des enjeux sociaux, �conomiques, environnementaux, industriels \ldots Donner si possible des arguments chiffr�s, par exemple, pertinence et port�e du projet par rapport � la demande �conomique (analyse du march�, analyse des tendances), analyse de la concurrence, indicateurs de r�duction de co�ts, perspectives de march�s (champs d?application,  \ldots). Indicateurs des gains environnementaux, cycle de vie \ldots}

\masubsection{Positionnement du projet}{Position of the project}\label{sec:position}
\Mycomment{Pr�ciser :
\begin{itemize}
\item 	positionnement du projet par rapport au contexte d�velopp� pr�c�demment : vis-�-vis des projets et recherches concurrents, compl�mentaires ou ant�rieurs, des brevets et standards \ldots
\item	indiquer si le projet s'inscrit dans la continuit� de projet(s) ant�rieurs d�j� financ�s par l'ANR. Dans ce cas, pr�senter bri�vement les r�sultats acquis,
\item	positionnement du projet par rapport aux axes th�matiques de l'appel � projets,
\item	positionnement du projet aux niveaux europ�en et international.
\end{itemize}
}

\subsubsection{Les contributions du projet (� discuter)}

\begin{itemize}
\item 	Apport m�thodologique combinant validation, v�rification et simulation
\item 	G�n�ration de test � partir de mod�les SysML
\item 	Formalisation et v�rification de propri�t�s fonctionnelles et non-fonctionnelles sur des mod�les SysML
\item 	Formalisation et v�rification de propri�t�s sur des mod�les Hardware
\item 	Tra�abilit� entre besoins SysML et Hardware  
\end{itemize}

\subsubsection{Impact  sur les �tudes}

\begin{enumerate}
\item 	r�duction  de la consommation d'�nergie pour fournir une r�ponse �conomique et  une grande stabilit� pour des grandes vitesses (contribuer � l'environnement).
\item Deux exemples connus dans ce domaine : r�duction de la tra�n�e a�rodynamique pour le corps d'Ahmed et  les trains � grande vitesse dans un tunnel de Bombardier.
\end{enumerate}

\masubsection{�tat de l'art}{State of the art}\label{sec:etat}
%\input{}

\masubsection{Objectifs et caract�re ambitieux/novateur du projet }{Objectives, originality and novelty of the project}\label{sec:objectifs}
%\input{}

%===========================================================

%===========================================================
\masection{Programme scientifique et technique, organisation du projet}{Scientific and technical programme, Project organisation}\label{sec:organisation}
%===========================================================
\redac{LIFC}
\masubsection{Programme scientifique et structuration du projet}{Scientific programme, project structure}\label{sec:struct}
\Mycomment{Pr�sentez le programme scientifique et justifiez la d�composition en t�ches du programme de travail en coh�rence avec les objectifs poursuivis. 
Utilisez un diagramme pour pr�senter les liens entre les diff�rentes t�ches (organigramme technique)
Les t�ches repr�sentent les grandes phases du projet. Elles sont en nombre limit�.
Le cas �ch�ant (programmes exigeant la pluridisciplinarit�), d�montrer l'articulation entre les disciplines scientifiques.
N'oubliez pas les activit�s et actions correspondant � la diss�mination et � la valorisation.
}



%===========================================================
\begin{figure}[htb]\begin{center} %   Figure taches
%===========================================================
\shabox{\includegraphics[width=11cm]{img/taches}}%
\caption{Encha�nement des t�ches}\label{fig:taches}
\end{center}\end{figure}
%===========================================================

Les diff�rentes �tapes  (cf. \fig{taches}) :

\begin{enumerate}
\item La premi�re �tape de ce projet consiste tout d'abord � r�diger le cahier des charges des �tudes de cas choisis.  Prendre en compte les diff�rents aspects quantitatifs des syst�mes � r�aliser ainsi l'expression des exigences. 

\item L'�tape suivante consiste � mod�liser ces syst�mes en SysML et formaliser les exigences en propri�t�s. Les propri�t�s peuvent �tre deux types et sont compl�mentaires : fonctionnelles et non fonctionnelles (�nergie, puissance, vitesse, pression, temps de r�action ?)

\item A partir de ces mod�les, identifier les propri�t�s qui peuvent �tre v�rifi�es sur les mod�les SysML (travaux sur les observateurs en passant par des formats interm�diaires)

\item Traduction des mod�les SysML en mod�les VHDL-AMS en utilisant des r�gles de transformations ad�quates

\item Simulation du mod�le VHDL-AMS  � l'aide d'un outil de simulation (Cadence, Smash). Cette �tape a pour objectif d'analyser le comportement du syst�me global et r�cup�rer certaines informations dans le but de reformuler les mod�les SysML et les propri�t�s selon certains param�tres

\item Expression des (certaines) exigences en langage PSL (utilisation des �l�ments des exigences + �l�ments de mod�le SysML) qui peuvent �tre compl�mentaires � celles v�rifi�es au niveau SysML 

\item V�rification des mod�les VHDL-AMS � l'aide de model checkers de propri�t�s (temporelles) d�crites en PSL

\item G�n�ration des cas de test � partir de mod�les SysML et/ou guid�e par les propri�t�s

\item Validation par cas de test et confrontation avec le mod�le physique 

\item Retour d'exp�riences sur les diff�rents r�sultats obtenus sur les mod�les propos�s
\end{enumerate}


\masubsection{Description des travaux par t�che }{Description by task}\label{sec:tasks}
\Mycomment{Pour chaque t�che, d�crire : 
\begin{itemize}
\item   les objectifs et �ventuels indicateurs de succ�s,
\item   le responsable et les partenaires impliqu�s (possibilit� de l'indiquer sous forme graphique),
\item   le programme d�taill� des travaux,
\item   les livrables,
\item   les contributions des partenaires (le � qui fait quoi �),
\item   la description des m�thodes et des choix techniques et de la mani�re dont les solutions seront apport�es,
\item   les risques et les solutions de repli envisag�es.
\end{itemize}
}


Certaines �tapes sont ind�pendantes et peuvent �tre trait�es en parall�le. On les retrouve dans les Workpackage suivants :
\begin{description}
\item[WP1] Mangement du projet et r�daction du cahier des charges
\item[WP2] Mod�lisation SysML des �tudes de cas et de leurs exigences 
\item[WP3] Transformation de mod�les SysML et des exigences vers le Hardware
\item[WP4] Validation et tests 
\item[WP5] Choix de la palte-forme et impl�mentions
\item[WP6] Validation des r�sultats obtenus dans les �tapes pr�c�dentes dans le contexte r�el
\end{description}

%------------------------------------------ WP1 -------------------------------
\subsubsection*{Workpackage 1: Management}
\begin{description}
\item[Leader:] \irit{} (Jean-Michel Bruel)
\item[Participants:] All
\item[Start:] T0
\item[End:] T0+40
\item[Description:] Manage project (communication, tasks and deliverables tracking)
\end{description}

%------------------------------------------ 1.1 -------------------------------
\begin{description}
\item[Task 1.1] -- Web Site  
\item[Leader:] ???
\item[Participants:] All
\item[Deliverable:]
	\begin{itemize}
	\item Web Site T0+3
	\end{itemize}
\item[Description:] Creation and animation of website
\end{description}
%------------------------------------------ 1.2 -------------------------------
\begin{description}
\item[Task 1.2] --  Tracking
\item[Leader:] \irit{} (Jean-Michel Bruel)
\item[Participants:] Board
\item[Deliverable:]
	\begin{itemize}
	\item Activity report: T0+12, T0+24 and T0+40
	\end{itemize}
\item[Description:] Management of project and respect of scheduling of tasks and deliverables.
\end{description}

%------------------------------------------ WP2 -------------------------------
\subsubsection*{Workpackage 2: Requirements and SysML Modeling}
\begin{description}
\item[Leader:] IRIT (Jean-Michel Bruel)
\item[Participants:] All
\item[Start:] T0
\item[End:] T0+40
\item[Description:] ce sous-projet a pour but d'�tablir le mod�le SysML. Il sera instanti� sur les �tudes de cas. Il permet de d�crire les exigences et les vues statiques et dynamiques de l'�tude de cas. Transition 1 
\end{description}

%------------------------------------------ 2.1 -------------------------------
\begin{description}
\item[Task 2.1] -- Definition and formalization of SysML kernel 
\item[Leader:] IRIT
\item[Participants:] INESS
\item[Begin:] T0
\item[End:] T0+12
\item[Deliverable:]
	\begin{itemize}
	\item Guideline for CoSyS modeling approach (including Meta Model SysML)
	\end{itemize}
\item[Description:]
\end{description}
%------------------------------------------ 2.2 -------------------------------
\begin{description}
\item[Task 2.2] --  Language for requirements modeling
\item[Leader:] INESS
\item[Participants:] 
\item[Begin:] T0
\item[End:] T0+12
\item[Deliverable:]
	\begin{itemize}
	\item State of the art, needs (-> criteria) and choice 
	\end{itemize}
\item[Description:] Study of the state of the art and choice for a language to formalize requirements into the SysML requirement Diagram. The idea is to used possibility to associate requirement with the elements of SysML (it is others diagrams that the requirement Diagram) and formalize requirements with a language as PSL, ROSETTA, \ldots
Est-ce que l'on parle des automates observateurs ?

\end{description}
%------------------------------------------ 2.3 -------------------------------
\begin{description}
\item[Task 2.3] --  Modeling for case study 1 
\item[Leader:] \irit{} can lead be can't ensure the modeling itself
\item[Participants:] 
\item[Begin:] T0+12
\item[End:] T0+24
\item[Deliverable:]
	\begin{itemize}
	\item Model of case study 1
	\end{itemize}
\item[Description:] Realization of case study model (SySML + Requirements).
\end{description}
%------------------------------------------ 2.4 -------------------------------
\begin{description}
\item[Task 2.4] --  Modeling for case study 2
\item[Leader:] INESS
\item[Participants:] 
\item[Begin:] T0+12
\item[End:] T0+24
\item[Deliverable:]
	\begin{itemize}
	\item Model of case study 2
	\end{itemize}
\item[Description:] Realization of case study model (SySML + Requirements).
\end{description}

%------------------------------------------ WP3 -------------------------------
\subsubsection*{Workpackage 3: Model Transformation}
\begin{description}
\item[Leader:] LISI (???)
\item[Participants:] LIFC
\item[Start:] T0+6
\item[End:] T0+24
\item[Description:] ce sous-projet a trois objectifs : i) La d�finition des transformations SysML vers une langage hardware (VHDL-AMS, SystemC, Verilog). ii) La mise en place des outils de transformation. iii) L'utilisation des exigences pour d�finir les propri�t�s hardware (PSL ou Rosetta) � v�rifier. Transitions 3 et 4
\end{description}
%------------------------------------------ 3.1 -------------------------------
\begin{description}
\item[Task 3.1] --  Definition of translation rules of SysML model into Hardware models 
\item[Leader:] LISI
\item[Participants:] 
\item[Begin:] T0+12
\item[End:] T0+24
\item[Deliverable:]
	\begin{itemize}
	\item Rules
	\end{itemize}
\item[Description:]
\end{description}
%------------------------------------------ 3.2 -------------------------------
\begin{description}
\item[Task 3.2] --  Definition of translation rules of SysML into Hardware properties
\item[Leader:] LIFC
\item[Participants:] 
\item[Begin:] T0+12
\item[End:] T0+30
\item[Deliverable:]
	\begin{itemize}
	\item Rules
	\end{itemize}
\item[Description:] this task will use information of Requirement Diagram and element linked into the others SysML diagram to produce Hardware properties
\end{description}

%------------------------------------------ WP4 -------------------------------
\subsubsection*{Workpackage 4: Verification, Simulation and Test generation}
\begin{description}
\item[Leader:] LIFC (Fabrice Bouquet)
\item[Participants:] INESS, IRIT, SMA
\item[Start:] T0+6
\item[End:] T0+40
\item[Description:] Transitions 2, 5, 6 et 11
\end{description}
%------------------------------------------ 4.1 -------------------------------
\begin{description}
\item[Task 4.1] -- Verification of SysML Model  
\item[Leader:] IRIT
\item[Participants:] LIFC
\item[Begin:] T0+12
\item[End:] T0+40
\item[Deliverable:]
	\begin{itemize}
	\item Component for platform + Results of verification on the case studies
	\end{itemize}
\item[Description:] This task defines and realizes a software component dedicated to realize verification on the SysML model. The first step is to define which kind of verification can be done.
\end{description}
%------------------------------------------ 4.2 -------------------------------
\begin{description}
\item[Task 4.2] -- Verification of properties on hardware models  
\item[Leader:] INESS
\item[Participants:] LIFC
\item[Begin:] T0+12
\item[End:] T0+30
\item[Deliverable:]
	\begin{itemize}
	\item Rules
	\end{itemize}
\item[Description:] This task defines and realizes a software component dedicated to realize verification on the Hardware models. The first step is to define which kind of verification can be done.
\end{description}
%------------------------------------------ 4.3 -------------------------------
\begin{description}
\item[Task 4.3] -- Component for test generation from SysML  
\item[Leader:] SMA
\item[Participants:] LIFC
\item[Begin:] T0+6
\item[End:] T0+36
\item[Deliverable:]
	\begin{itemize}
	\item Description of test generation component for platform
	\end{itemize}
\item[Description:] Definition of dedicated algorithms for test generation based on SysML model and formalized requirements. Design and implement of components for test generation.
\end{description}
%------------------------------------------ 4.4 -------------------------------
\begin{description}
\item[Task 4.4] --  Simulation of the hardware model driven by tests 
\item[Leader:] LIFC
\item[Participants:] SMA
\item[Begin:] T0+24
\item[End:] T0+36
\item[Deliverable:]
	\begin{itemize}
	\item Component
	\end{itemize}
\item[Description:] Realization of test publisher for hardware models simulation tools as Cadence or Smash.
\end{description}
%------------------------------------------ 4.5 -------------------------------
\begin{description}
\item[Task 4.5] -- Test execution into case studies 
\item[Leader:] LIFC
\item[Participants:] SMA
\item[Begin:] T0+24
\item[End:] T0+36
\item[Deliverable:]
	\begin{itemize}
	\item Component
	\end{itemize}
\item[Description:] Realization of test publisher for implementation
\end{description}

%------------------------------------------ WP5 -------------------------------
\subsubsection*{Workpackage 5: Implementation of platform for micro-system development based on Papyrus/Eclipse}
\begin{description}
\item[Leader:] LIFC (Fabrice Ambert)
\item[Participants:] SMA
\item[Start:] T0
\item[End:] T0+40
\item[Description:] Transitions 1 � 6 et 11
\end{description}
%------------------------------------------ 5.1 -------------------------------
\begin{description}
\item[Task 5.1] -- Architecture and technological choice for CoSyS Platform  
\item[Leader:] SMA
\item[Participants:] LIFC
\item[Begin:] T0
\item[End:] T0+6
\item[Deliverable:]
	\begin{itemize}
	\item Guide and repository for Platform
	\end{itemize}
\item[Description:] Choix technologique de l'environnement
\end{description}
%------------------------------------------ 5.2 -------------------------------
\begin{description}
\item[Task 5.2] --  Integration of developed modules 
\item[Leader:] LIFC
\item[Participants:] SMA
\item[Begin:] T0+6
\item[End:] T0+40
\item[Deliverable:] 
	\begin{itemize}
	\item Platform
	\end{itemize}
\item[Description:] Realization of open source platform
\end{description}
%------------------------------------------ 5.3 -------------------------------
\begin{description}
\item[Task 5.3] --  Platform Validation
\item[Leader:] FEMTO
\item[Participants:] LIFC, SMA
\item[Begin:] T0+24
\item[End:] T0+40
\item[Deliverable:]
	\begin{itemize}
	\item Feedback of platform uses
	\end{itemize}
\item[Description:]
\end{description}

%------------------------------------------ WP6 -------------------------------
\subsubsection*{Workpackage 6: Validation in the real context}
\begin{description}
\item[Leader:] ??? (Indus)
\item[Participants:] INESS, FEMTO-ST
\item[Start:] T0
\item[End:] T0+40
\item[Description:] Transitions 7,8 et 9
\end{description}
%------------------------------------------ 6.1 -------------------------------
\begin{description}
\item[Task 6.1] -- Implementation of 1st case study 
\item[Leader:] \femto
\item[Participants:] 
\item[Begin:] T0
\item[End:] T0+24
\item[Deliverable:]
	\begin{itemize}
	\item Real System
	\end{itemize}
\item[Description:]
realization of real system depicted in case study 1. 2 steps for case study implementation:
\begin{itemize}
\item Prototyping to tune parameters for SysML model of system
\item Back to Back implementation and validation of implementation with test generation (transition 7)
\item Synthetization of implementation from hardware models (transitions 8 and 9)
\end{itemize}
\end{description} 

%------------------------------------------ 6.2 -------------------------------
\begin{description}
\item[Task 6.2] --  Consistency between real system and simulation for 1st case study
\item[Leader:] \femto
\item[Participants:] 
\item[Begin:] T0+24
\item[End:] T0+30
\item[Deliverable:]
	\begin{itemize}
	\item ??
	\end{itemize}
\item[Description:]
\end{description}
%------------------------------------------ 6.3 -------------------------------
\begin{description}
\item[Task 6.3] --  Implementation of 2nd case study
\item[Leader:] Indus
\item[Participants:] 
\item[Begin:] T0
\item[End:] T0+32
\item[Deliverable:]
	\begin{itemize}
	\item Real System
	\end{itemize}
\item[Description:] realization of real system depicted in case study
\end{description}
%------------------------------------------ 6.4 -------------------------------
\begin{description}
\item[Task 6.4] --  Consistency between real system and simulation for 2nd case study
\item[Leader:] Indus
\item[Participants:] 
\item[Begin:] T0+32
\item[End:] T0+40
\item[Deliverable:]
	\begin{itemize}
	\item ??
	\end{itemize}
\item[Description:]
\end{description}
%------------------------------------------ 6.5 -------------------------------
\begin{description}
\item[Task 6.5] --  Evaluation of process
\item[Leader:] Indus
\item[Participants:] 
\item[Begin:] T0+36
\item[End:] T0+40
\item[Deliverable:]
	\begin{itemize}
	\item Platform
	\end{itemize}
\item[Description:]
\end{description}

\Mycomment{A discuter :
\begin{enumerate}
\item Quels apports scientifiques pour chacun des partenaires ?
\item Etat de l'art sur chacune des transitions (voir dessin) :
\begin{itemize}
\item Sur les travaux de la v�rification formelle au niveau SysML
\item Travaux sur les transformations de mod�les vers des mod�les Hardware
\item G�n�ration de test � partir de mod�les SysML 
\item Les logiques d'expression de propri�t�s non fonctionnelles
\item Les techniques de v�rification de mod�les hardware
\end{itemize}
\end{enumerate}
}

\masubsection{Management du projet }{Project management}\label{sec:management}
\redac{JMB}
\Mycomment{Pr�ciser les aspects organisationnels du projet et les modalit�s de coordination (si possible individualisation d'une t�che de coordination).}

A Board and project coordinator manages the project. Scientific leader of each partner composes the board and the project coordinator:

\begin{itemize}
\item Project coordinator: J.-M. Bruel
\item Partner leaders: 
\begin{description}
\item[FEMTO :] J.-F. Manceau
\item[IRIT :] Iulian Ober
\item[INESS :] Yves-Andr� Chapuis
\item[LIFC :] Hassan Mountassir
\item[LISI :] XXX
\item[SMA :] Bruno Legeard
\item[Indus :] XXX
\end{description}
\end{itemize}

\masubsection{Calendrier des t�ches, livrables et jalons }{Tasks schedule, deliverables and milestones}\label{sec:gantt}
\Mycomment{Pr�senter sous forme graphique un �ch�ancier des diff�rentes t�ches et leurs d�pendances (diagramme de Gantt par exemple).
Pr�senter un tableau synth�tique de l'ensemble des livrables du projet (num�ro de t�che, date, intitul�, responsable).
Pr�ciser de fa�on synth�tique les jalons scientifiques et/ou techniques, les principaux points de rendez-vous, les points bloquants ou al�as qui risquent de remettre en cause l'aboutissement du projet ainsi que les r�unions de projet pr�vues.
}
%===========================================================

%===========================================================
\masection{Strat�gie de valorisation, de protection et d?exploitation des r�sultats }{Dissemination and exploitation of results. intellectual property}\label{sec:valo}
%===========================================================
\input{.tex}
%===========================================================

%===========================================================
\masection{Description du partenariat }{Consortium description }\label{sec:consortium}
%===========================================================
\input{.tex}
%===========================================================

%===========================================================
\masection{Justification scientifique des moyens demand�s}{Scientific justification of requested ressources}\label{sec:moyens}
%===========================================================
\input{.tex}
%===========================================================

%===========================================================
\masection{Annexes}{Annexes}\label{sec:annexes}
%===========================================================
\input{.tex}
%===========================================================

\bibliographystyle{alpha}
%\bibliography{cosys}
% ficher .bib a fournir!

\end{document}
