%====================================
% Titre                 : Exemple de rapport de projet
% Last Update           : 25/01/2011
% 						: first draft taken from the CPP project (http://www.lix.polytechnique.fr/~bouissou/cpp/index.php?n=Main.Documents)
%====================================
\documentclass[oneside]{article} % or report
%====================================
\usepackage{a4wide}
\usepackage[francais,public]{anr}
% options:
% lang: francais ou english (defaut: francais)
% classification: public, restricted, confidential (defaut: public).
\usepackage[french]{babel}
\usepackage[latin1]{inputenc}
\usepackage{latexsym,times}
\usepackage{tabularx}
\usepackage[sc]{mathpazo}
\linespread{1.05}         % Palatino needs more leading (space between lines)

\newcommand\Mycomment[1]%          %  temporary remark for the
 {\par\smallskip                     %  authors:
  \begin{center}%                    %
   \fbox%                            %    --------
   {\parbox{0.9\linewidth}%          %    |  #1  |
    {\raggedright--- 
	\textcolor{red}{Note :} #1}%     %    --------
   }%                                %
  \end{center}%                      %
  \par\smallskip                     %
 }
%\renewcommand\Mycomment[1]{}
\date{\today}
\author{Jean-Michel Bruel}
\project{MACAO, IRIT, CNRS}
\reference{0.0}
\title{Un exemple de rapport de projet}
\version{1}
\status{Draft}

%--------------------------- Notes (si besoin) ---------------------------
%\begin{notes} % used with \showNotes
%Ceci est une note
%\end{notes}
%-------------------------------------------------------------------------

%===========================================================
\begin{document}
\pagestyle{fancy}
%\showNotes
\makecosys
~\\\vspace{1cm}
\tableofcontents
%===========================================================
\Mycomment{les encadr\'es comme celui-ci sont produits avec une
  commande \verb:comment:, son but est d'inclure des commentaires pour les
  co-auteurs du document.  Ils doivent dispara\^{\i}tre dans la version finale
  du document. Tous commentaires bienvenus!}

%===========================================================
\masection{R\'esum\'e de la proposition}{Executive summary}\label{sec:intro}
%===========================================================
\Mycomment{Recopier le r�sum� utilis� dans le document administratif et financier (dit document de soumission)}

%===========================================================

%===========================================================
\masection{Contexte, positionnement et objectifs de la proposition}{Context, position and objectives of the proposal}\label{sec:contexte}
%===========================================================
\Mycomment{A titre indicatif : de 5 � 10 pages pour ce chapitre.

Pr�sentation g�n�rale du probl�me qu'il est propos� de traiter dans le projet et du cadre de travail (recherche fondamentale, industrielle ou d�veloppement exp�rimental).}

\masubsection{Contexte et enjeux �conomiques et soci�taux}{Context, social and economic issues}\label{sec:enjeuxEco}
\Mycomment{D�crire le contexte �conomique, social, r�glementaire \ldots dans lequel se situe le projet en pr�sentant une analyse des enjeux sociaux, �conomiques, environnementaux, industriels \ldots Donner si possible des arguments chiffr�s, par exemple, pertinence et port�e du projet par rapport � la demande �conomique (analyse du march�, analyse des tendances), analyse de la concurrence, indicateurs de r�duction de co�ts, perspectives de march�s (champs d?application,  \ldots). Indicateurs des gains environnementaux, cycle de vie \ldots}

\masubsection{Positionnement du projet}{Position of the project}\label{sec:position}
\Mycomment{Pr�ciser :
\begin{itemize}
\item 	positionnement du projet par rapport au contexte d�velopp� pr�c�demment : vis-�-vis des projets et recherches concurrents, compl�mentaires ou ant�rieurs, des brevets et standards \ldots
\item	indiquer si le projet s'inscrit dans la continuit� de projet(s) ant�rieurs d�j� financ�s par l'ANR. Dans ce cas, pr�senter bri�vement les r�sultats acquis,
\item	positionnement du projet par rapport aux axes th�matiques de l'appel � projets,
\item	positionnement du projet aux niveaux europ�en et international.
\end{itemize}
}

\subsubsection{Les contributions du projet (� discuter)}

\begin{itemize}
\item 	Apport m�thodologique combinant validation, v�rification et simulation
\item 	G�n�ration de test � partir de mod�les SysML
\item 	Formalisation et v�rification de propri�t�s fonctionnelles et non-fonctionnelles sur des mod�les SysML
\item 	Formalisation et v�rification de propri�t�s sur des mod�les Hardware
\item 	Tra�abilit� entre besoins SysML et Hardware  
\end{itemize}

\subsubsection{Impact  sur les �tudes}

\begin{enumerate}
\item 	r�duction  de la consommation d'�nergie pour fournir une r�ponse �conomique et  une grande stabilit� pour des grandes vitesses (contribuer � l'environnement).
\item Deux exemples connus dans ce domaine : r�duction de la tra�n�e a�rodynamique pour le corps d'Ahmed et  les trains � grande vitesse dans un tunnel de Bombardier.
\end{enumerate}

\masubsection{�tat de l'art}{State of the art}\label{sec:etat}
%\input{}

\masubsection{Objectifs et caract�re ambitieux/novateur du projet }{Objectives, originality and novelty of the project}\label{sec:objectifs}
%\input{}

%===========================================================

%===========================================================
\masection{Programme scientifique et technique, organisation du projet}{Scientific and technical programme, Project organisation}\label{sec:organisation}
%===========================================================
%\input{}
%===========================================================

%===========================================================
\masection{Strat�gie de valorisation, de protection et d'exploitation des r�sultats }{Dissemination and exploitation of results. Intellectual property}\label{sec:valo}
%===========================================================
\input{.tex}
%===========================================================

%===========================================================
\masection{Description du partenariat }{Consortium description }\label{sec:consortium}
%===========================================================
\input{.tex}
%===========================================================

%===========================================================
\masection{Justification scientifique des moyens demand�s}{Scientific justification of requested ressources}\label{sec:moyens}
%===========================================================
\input{.tex}
%===========================================================

%===========================================================
\masection{Annexes}{Annexes}\label{sec:annexes}
%===========================================================
\input{.tex}
%===========================================================

\bibliographystyle{alpha}
%\bibliography{cosys}
% ficher .bib a fournir!

\end{document}
