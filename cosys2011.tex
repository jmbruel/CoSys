%====================================
% Titre                 : Exemple de rapport de projet
% Last Update           : 25/01/2011
% 						: first draft taken from the CPP project (http://www.lix.polytechnique.fr/~bouissou/cpp/index.php?n=Main.Documents)
%====================================
\documentclass[oneside]{article} % or report
%====================================
\usepackage{a4wide}
\usepackage[francais,public]{anr}
% options:
% lang: francais ou english (defaut: francais)
% classification: public, restricted, confidential (defaut: public).
\usepackage[french]{babel}
\usepackage[latin1]{inputenc}
\usepackage{latexsym,times}
\usepackage{tabularx}
\usepackage[sc]{mathpazo}
\linespread{1.05}         % Palatino needs more leading (space between lines)

\newcommand\Mycomment[1]%          %  temporary remark for the
 {\par\smallskip                     %  authors:
  \begin{center}%                    %
   \fbox%                            %    --------
   {\parbox{0.9\linewidth}%          %    |  #1  |
    {\raggedright--- 
	\textcolor{red}{Note :} #1}%     %    --------
   }%                                %
  \end{center}%                      %
  \par\smallskip                     %
 }
%\renewcommand\Mycomment[1]{}
\date{\today}
\author{Jean-Michel Bruel}
\project{MACAO, IRIT, CNRS}
\reference{0.0}
\title{Un exemple de rapport de projet}
\version{1}
\status{Draft}

%--------------------------- Notes (si besoin) ---------------------------
%\begin{notes} % used with \showNotes
%Ceci est une note
%\end{notes}
%-------------------------------------------------------------------------

%===========================================================
\begin{document}
\pagestyle{fancy}
%\showNotes
\makecosys
~\\\vspace{1cm}
\tableofcontents
%===========================================================
\Mycomment{
Avant de soumettre ce document :
\begin{itemize}
\item   Supprimer tous les commentaires en rempla�ant la commande \verb:Mycomment:
\item   Mettre la table des mati�res � jour (double compilation). 
\item   Donner toutes les r�f�rences bibliographiques en annexe (cf. \masec{refs}).
\item   Ce document, hors annexes, ne doit pas d�passer 40 pages, corps de texte en police de taille 11. \underline{Ce point constitue un crit�re de recevabilit� de la proposition de projet}.
\end{itemize}
}

%===========================================================
\masection{R\'esum\'e de la proposition}{Executive summary}\label{sec:intro}
%===========================================================
\Mycomment{Recopier le r�sum� utilis� dans le document administratif et financier (dit document de soumission)}

%===========================================================

%===========================================================
\masection{Contexte, positionnement et objectifs de la proposition}{Context, position and objectives of the proposal}\label{sec:contexte}
%===========================================================
\Mycomment{A titre indicatif : de 5 � 10 pages pour ce chapitre.

Pr�sentation g�n�rale du probl�me qu'il est propos� de traiter dans le projet et du cadre de travail (recherche fondamentale, industrielle ou d�veloppement exp�rimental).}

\masubsection{Contexte et enjeux �conomiques et soci�taux}{Context, social and economic issues}\label{sec:enjeuxEco}
\Mycomment{D�crire le contexte �conomique, social, r�glementaire \ldots dans lequel se situe le projet en pr�sentant une analyse des enjeux sociaux, �conomiques, environnementaux, industriels \ldots Donner si possible des arguments chiffr�s, par exemple, pertinence et port�e du projet par rapport � la demande �conomique (analyse du march�, analyse des tendances), analyse de la concurrence, indicateurs de r�duction de co�ts, perspectives de march�s (champs d?application,  \ldots). Indicateurs des gains environnementaux, cycle de vie \ldots}

\masubsection{Positionnement du projet}{Position of the project}\label{sec:position}
\Mycomment{Pr�ciser :
\begin{itemize}
\item 	positionnement du projet par rapport au contexte d�velopp� pr�c�demment : vis-�-vis des projets et recherches concurrents, compl�mentaires ou ant�rieurs, des brevets et standards \ldots
\item	indiquer si le projet s'inscrit dans la continuit� de projet(s) ant�rieurs d�j� financ�s par l'ANR. Dans ce cas, pr�senter bri�vement les r�sultats acquis,
\item	positionnement du projet par rapport aux axes th�matiques de l'appel � projets,
\item	positionnement du projet aux niveaux europ�en et international.
\end{itemize}
}

\subsubsection{Les contributions du projet (� discuter)}

\begin{itemize}
\item 	Apport m�thodologique combinant validation, v�rification et simulation
\item 	G�n�ration de test � partir de mod�les SysML
\item 	Formalisation et v�rification de propri�t�s fonctionnelles et non-fonctionnelles sur des mod�les SysML
\item 	Formalisation et v�rification de propri�t�s sur des mod�les Hardware
\item 	Tra�abilit� entre besoins SysML et Hardware  
\end{itemize}

\subsubsection{Impact  sur les �tudes}

\begin{enumerate}
\item 	r�duction  de la consommation d'�nergie pour fournir une r�ponse �conomique et  une grande stabilit� pour des grandes vitesses (contribuer � l'environnement).
\item Deux exemples connus dans ce domaine : r�duction de la tra�n�e a�rodynamique pour le corps d'Ahmed et  les trains � grande vitesse dans un tunnel de Bombardier.
\end{enumerate}

\masubsection{�tat de l'art}{State of the art}\label{sec:etat}
%\input{}

\masubsection{Objectifs et caract�re ambitieux/novateur du projet }{Objectives, originality and novelty of the project}\label{sec:objectifs}
%\input{}

%===========================================================

%===========================================================
\masection{Programme scientifique et technique, organisation du projet}{Scientific and technical program, Project organisation}\label{sec:organisation}
%===========================================================

\Mycomment{A titre indicatif : de 8 � 12 pages pour ce chapitre, en fonction du nombre de t�ches.}
\masubsection{Programme scientifique et structuration du projet}{Scientific programme, project structure}\label{sec:struct}
\Mycomment{Pr�sentez le programme scientifique et justifiez la d�composition en t�ches du programme de travail en coh�rence avec les objectifs poursuivis. 
Utilisez un diagramme pour pr�senter les liens entre les diff�rentes t�ches (organigramme technique)
Les t�ches repr�sentent les grandes phases du projet. Elles sont en nombre limit�.
Le cas �ch�ant (programmes exigeant la pluridisciplinarit�), d�montrer l'articulation entre les disciplines scientifiques.
N'oubliez pas les activit�s et actions correspondant � la diss�mination et � la valorisation.
}

\masubsection{Management du projet }{Project management}\label{sec:management}
\Mycomment{Pr�ciser les aspects organisationnels du projet et les modalit�s de coordination (si possible individualisation d'une t�che de coordination).}

\masubsection{Description des travaux par t�che }{Description by task}\label{sec:tasks}
\Mycomment{Pour chaque t�che, d�crire : 
\begin{itemize}
\item   les objectifs et �ventuels indicateurs de succ�s,
\item   le responsable et les partenaires impliqu�s (possibilit� de l'indiquer sous forme graphique),
\item   le programme d�taill� des travaux,
\item   les livrables,
\item   les contributions des partenaires (le � qui fait quoi �),
\item   la description des m�thodes et des choix techniques et de la mani�re dont les solutions seront apport�es,
\item   les risques et les solutions de repli envisag�es.
\end{itemize}
}

\masubsubsection{T�che 1}{Task 1}
\masubsubsection{T�che 2}{Task 2}
\masubsubsection{T�che 3}{Task 3}

\masubsection{Calendrier des t�ches, livrables et jalons }{Tasks schedule, deliverables and milestones}\label{sec:gantt}
\Mycomment{Pr�senter sous forme graphique un �ch�ancier des diff�rentes t�ches et leurs d�pendances (diagramme de Gantt par exemple).
Pr�senter un tableau synth�tique de l'ensemble des livrables du projet (num�ro de t�che, date, intitul�, responsable).
Pr�ciser de fa�on synth�tique les jalons scientifiques et/ou techniques, les principaux points de rendez-vous, les points bloquants ou al�as qui risquent de remettre en cause l'aboutissement du projet ainsi que les r�unions de projet pr�vues.
}
%===========================================================

%===========================================================
\masection{Strat�gie de valorisation, de protection et d'exploitation des r�sultats }{Dissemination and exploitation of results. Intellectual property}\label{sec:valo}
%===========================================================
\Mycomment{A titre indicatif : 2 pages pour ce chapitre.
Pr�senter les strat�gies de valorisation des r�sultats :
\begin{itemize}
\item   la communication scientifique,
\item   la communication aupr�s du grand public (un budget sp�cifique peut �tre pr�vu),
\item   la valorisation des r�sultats attendus,
\item   les retomb�es scientifiques, techniques, industrielles, �conomiques, \ldots
\item   la place du projet dans la strat�gie industrielle des entreprises partenaires du projet,
\item   autres retomb�es (normalisation, information des pouvoirs publics, ...),
\item   les �ch�ances et la nature des retomb�es technico- �conomiques attendues,
\item   l'incidence �ventuelle sur l'emploi, la cr�ation d'activit�s nouvelles, \ldots
\end{itemize}

Pr�senter les grandes lignes des modes de protection et d'exploitation des r�sultats
Pour les projets partenariaux organismes de recherche/entreprises, les partenaires devront conclure, sous l'�gide du coordinateur du projet, un accord de consortium dans un d�lai de un an si le projet est retenu pour financement.  
Pour les projets acad�miques, l'accord de consortium n'est pas obligatoire mais fortement conseill�.
}

%===========================================================

%===========================================================
\masection{Description du partenariat }{Consortium description }\label{sec:consortium}
%===========================================================
\Mycomment{A titre indicatif : de 2 � 5 pages pour ce chapitre, en fonction du nombre de partenaires}

\masubsection{Description, ad�quation et compl�mentarit� des partenaires }{Partners description \& relevance, complementarity}\label{sec:adeq}

\Mycomment{(Maximum 0,5 page par partenaire)
D�crire bri�vement chaque partenaire et fournir ici les �l�ments permettant d'appr�cier la qualification des partenaires dans le projet (le `` pourquoi qui fait quoi "). Il peut s'agir de r�alisations pass�es, d'indicateurs (publications, brevets), de l'int�r�t du partenaire pour le projet?

Montrer la compl�mentarit� et la valeur ajout�e des coop�rations entre les diff�rents partenaires. L'interdisciplinarit� et l'ouverture � diverses collaborations seront � justifier en accord avec les orientations du projet. (1 page maximum)
}

%===========================================================
\begin{center}\begin{minipage}[h]{\textwidth}\begin{center}
\begin{tabular}{|c|c|c|c|c|c|c|c|}
\hline  & \PartnOne & \PartnTwo & \PartnThree & \PartnFour & \PartnFive & \PartnSix & \PartnSeven \\ 
\hline 
\hline \PartnOne &  &  &  & X & X &  &  \\ 
\hline \PartnTwo &  &  &  &  &  &  &  \\ 
\hline \PartnThree &  &  &  &  & X &  &  \\ 
\hline \PartnFour &  &  &  &  & X &  & \\ 
\hline \PartnFive &  &  &  &  &  & X & \\ 
\hline \PartnSix &  &  &  &  &  &  &  \\ 
\hline \PartnSeven &  &  &  &  &  &  &  \\ 
\hline %=====================================================
\end{tabular}\end{center}\end{minipage}\end{center}
%===========================================================


\subsubsection{FEMTO-ST}
\femto{} : Mahmoud Addouche, Jean-Fran�ois Manceau et Reda Yahaoui. Ils travaillent sur la conception et la r�alisation de micro-actionneurs distribu�s. Ils sont int�ress�s pour apporter une �tude de cas.

\subsubsection{LIFC}
\lifc{} : Fabrice Bouquet, Jacques Julliand et Hassan Mountassir. Ils travaillent sur la g�n�ration test � partir de mod�les SysML. Ils sont int�ress�s par la transformation SysML vers VHDL-AMS, la g�n�ration de test � partir de SysML et PSL et la v�rification de propri�t�s PSL (peut �tre en interaction avec l'IRIT si au niveau SysML).

\subsubsection{LISI}
Du \lisi{}, Yamine Ait-Aimeur ne peut pas participer mais va proposer quelqu'un, ing�nieurie des mod�les. Il travaillerait sur la transformation des mod�les SysML.

\subsubsection{INESS}
\iness{} : Yves-Andr� Chapuis. Il travaille sur les aspects VHDL-AMS. Il serait int�ress� par l'apport SysML et l'utilisation de PSL pour g�n�rer et compl�ter les �l�ments VHDL-AMS.
\subsubsection{IRIT}
\irit{} : Jean-Michel Bruel, Iulian Ober, Samir Hameg. 

Atelier de mod�lisation SysML bas� sur Omega (d�velopp� dans le cadre du projet europ�en Omega pour un atelier g�nie logiciel UML). Ils apportent les comp�tences sur la mod�lisation SysML et la v�rification  au niveau SysML. Ils seraient int�ress�s sur ces th�mes et ajouter la v�rification pour les syst�mes embarqu�s de type SOC.

Projets en cours : 
\begin{itemize}
\item \fullmde{} (\url{http://www-verimag.imag.fr/Full-MDE.html})
\item \socket{} (\url{http://socket.imag.fr/})
\end{itemize}

\subsubsection{SMA}
%\input{SMA}

\masubsection{Qualification du coordinateur du projet }{Qualification of the project coordinator}\label{sec:coord}
\Mycomment{(0,5 page maximum).
Fournir les �l�ments permettant de juger la capacit� du coordinateur � coordonner le projet.}

\masubsection{Qualification, r�le et implication des participants }{Qualification and contribution of each partner}\label{sec:qualif}
\Mycomment{(2 pages maximum).
Qualifier les personnes, pr�ciser leurs activit�s principales  et leurs comp�tences propres. Pour chaque partenaire remplir le tableau ci-dessous
}

%===========================================================
\begin{center}\begin{minipage}[h]{\textwidth}\begin{center}
%\begin{tabular}{|p{18mm}|p{25mm}|r|p{15mm}|p{50mm}|p{50mm}}
\begin{tabular}{|l|l|l|l|l|l|l|}
\hline %=====================================================
\monlabel{Partenaire}{Partner} &
\monlabel{Nom}{Name} &
\monlabel{Pr�nom }{First name} &
\monlabel{Emploi actuel}{Position} &
\monlabel{Discipline }{Research Field} &
PM\footnote{\monlabel{Personne.mois sur la dur�e du projet.}{People.month for the project duration}}  & 
\monlabel{R�le}{Contribution} \\
% & & & & & & 4 lignes max  \\ 
\hline 

\hline \lifc & Bouquet & F. & Professor &  &  &  \\ 
\hline \irit & Bruel & J.-M. & Professor & MDE/SysML & 12 & Project Leader \\ 
\hline \irit & Iulian & O. & MC (HDR) & V\&V & 9 & SysML validation \\ 
\hline \irit & Hameg & S. & Ph.D. & CoSimulation & 9 & Simulation \\ 
\hline \ldots & \ldots & \ldots & \ldots & \ldots & \ldots & \ldots \\ 
%\hline  &  &  &  &  &  &  \\ 
%\hline 
\hline %=====================================================
\end{tabular}\end{center}\end{minipage}\end{center}
%===========================================================

\Mycomment{Pour chacune des personnes dont l'implication dans le projet est sup�rieure 
� 25\%\ de son temps sur la totalit� du projet (c'est-�-dire une moyenne de 3 hommes.mois par ann�e de projet), 
une biographie d'une page maximum sera plac�e en annexe (cf.~\masec{bio}) du pr�sent document qui comportera :
\begin{itemize}
\item Nom, pr�nom, �ge, cursus, situation actuelle
\item Autres exp�riences professionnelles
\item Liste des cinq publications (ou brevets) les plus significatives des cinq derni�res ann�es, nombre de publications dans les revues internationales ou actes de congr�s � comit� de lecture.
\item Prix, distinctions
\end{itemize}
Si besoin, pour chacune des personnes, leur implication dans d'autres projets (Contrats publics et priv�s effectu�s ou en cours sur les trois derni�res ann�es) sera pr�sent�e selon le mod�le fourni en annexe. Les tableaux seront plac�s en annexe  (cf.~\masec{implication}). On pr�cisera l'implication dans des projets europ�ens ou dans d'autres types de projets nationaux ou internationaux. Expliciter l'articulation entre les travaux propos�s et les travaux ant�rieurs ou d�j� en cours.
}
%===========================================================

%===========================================================
\masection{Justification scientifique des moyens demand�s}{Scientific justification of requested ressources}\label{sec:moyens}
%===========================================================
\Mycomment{On pr�sentera ici la justification scientifique et technique des moyens demand�s dans le document de soumission  par chaque partenaire et synth�tis�s � l'�chelle du projet dans la fiche ``Tableaux r�capitulatifs" du document administratif et financier (dit document de soumission) tel que rempli en ligne sur le site de soumission.

Chaque partenaire justifiera les moyens qu'il demande en distinguant les diff�rents postes de d�penses.
(Maximum 2 pages par partenaire)
}
%\masubsection{LIFC}{LIFC}\input{demande-LIFC.tex}
%\masubsection{LISI}{LISI}\input{demande-LISI.tex}
\masubsection{IRIT}{IRIT}\masubsubsection{�quipement}{Equipment}
\Mycomment{Pr�ciser la nature des �quipements et justifier le choix des �quipements
Si n�cessaire, pr�ciser la part de financement demand� sur le projet et si les achats envisag�s doivent �tre compl�t�s par d'autres sources de financement. Si tel est le cas, indiquer le montant et l'origine de ces financements compl�mentaires.
Un devis sera demand� si le projet est retenu pour financement.}

\masubsubsection{Personnel}{Staff}
\Mycomment{Le personnel non permanent (th�ses, post- doctorants, CDD...) financ� sur le projet devra �tre justifi�.
Fournir  les profils des postes � pourvoir pour les personnels � recruter.

Pour les th�ses, pr�ciser si des demandes de bourse de th�se sont pr�vues ou en cours, en pr�ciser la nature et la part de financement imputable au projet. }

\masubsubsection{Prestation de service externe}{Subcontracting}
\Mycomment{Pr�ciser :
\begin{itemize}
\item   la nature des prestations
\item   le type de prestataire.
\end{itemize}
}

\masubsubsection{Missions}{Travel}
\Mycomment{Pr�ciser :
\begin{itemize}
\item   les missions li�es aux travaux d'acquisition sur le terrain (campagnes de mesures?)
\item   les missions relevant de colloques, congr�s?
\end{itemize}
}
\masubsubsection{D�penses justifi�es sur une proc�dure de facturation interne}{Costs justified by internal procedures of invoicing}
\Mycomment{Pr�ciser la nature des prestations.}

\masubsubsection{Autres d�penses de fonctionnement}{Other expenses}
\Mycomment{Toute d�pense significative relevant de ce poste devra �tre justifi�e.}


\masubsection{...}{...}\masubsubsection{�quipement}{Equipment}
\Mycomment{Pr�ciser la nature des �quipements et justifier le choix des �quipements
Si n�cessaire, pr�ciser la part de financement demand� sur le projet et si les achats envisag�s doivent �tre compl�t�s par d?autres sources de financement. Si tel est le cas, indiquer le montant et l?origine de ces financements compl�mentaires.
Un devis sera demand� si le projet est retenu pour financement.}

\masubsubsection{Personnel}{Staff}
\Mycomment{Le personnel non permanent (th�ses, post- doctorants, CDD...) financ� sur le projet devra �tre justifi�.
Fournir  les profils des postes � pourvoir pour les personnels � recruter.

Pour les th�ses, pr�ciser si des demandes de bourse de th�se sont pr�vues ou en cours, en pr�ciser la nature et la part de financement imputable au projet. }

\masubsubsection{Prestation de service externe}{Subcontracting}
\Mycomment{Pr�ciser :
-	la nature des prestations
-	le type de prestataire.
}

\masubsubsection{Missions}{Travel}
\Mycomment{Pr�ciser :
-	les missions li�es aux travaux d?acquisition sur le terrain (campagnes de mesures?)
-	les missions relevant de colloques, congr�s?
}
\masubsubsection{D�penses justifi�es sur une proc�dure de facturation interne}{Costs justified by internal procedures of invoicing}
\Mycomment{Pr�ciser la nature des prestations.}

\masubsubsection{Autres d�penses de fonctionnement}{Other expenses}
\Mycomment{Toute d�pense significative relevant de ce poste devra �tre justifi�e.}



%===========================================================

%===========================================================
\masection{Annexes}{Annexes}\label{sec:annexes}
%===========================================================

\masubsection{R�f�rences bibliographiques }{References}\label{sec:refs}

\bibliographystyle{alpha}
%\bibliography{cosys}
% ficher .bib a fournir!

\masubsection{Biographies}{CV / Resume}\label{sec:bio}
\masubsection{Implication des personnes dans d'autres contrats}{Staff involvement in other contracts}\label{sec:implication} 

\end{document}
