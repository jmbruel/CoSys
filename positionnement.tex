\Mycomment{Pr�ciser :
\begin{itemize}
\item 	positionnement du projet par rapport au contexte d�velopp� pr�c�demment : vis-�-vis des projets et recherches concurrents, compl�mentaires ou ant�rieurs, des brevets et standards \ldots
\item	indiquer si le projet s'inscrit dans la continuit� de projet(s) ant�rieurs d�j� financ�s par l'ANR. Dans ce cas, pr�senter bri�vement les r�sultats acquis,
\item	positionnement du projet par rapport aux axes th�matiques de l'appel � projets,
\item	positionnement du projet aux niveaux europ�en et international.
\end{itemize}
}

\subsubsection{Les contributions du projet (� discuter)}

\begin{itemize}
\item 	Apport m�thodologique combinant validation, v�rification et simulation
\item 	G�n�ration de test � partir de mod�les SysML
\item 	Formalisation et v�rification de propri�t�s fonctionnelles et non-fonctionnelles sur des mod�les SysML
\item 	Formalisation et v�rification de propri�t�s sur des mod�les Hardware
\item 	Tra�abilit� entre besoins SysML et Hardware  
\end{itemize}

\subsubsection{Impact  sur les �tudes}

\begin{enumerate}
\item 	r�duction  de la consommation d'�nergie pour fournir une r�ponse �conomique et  une grande stabilit� pour des grandes vitesses (contribuer � l'environnement).
\item Deux exemples connus dans ce domaine : r�duction de la tra�n�e a�rodynamique pour le corps d'Ahmed et  les trains � grande vitesse dans un tunnel de Bombardier.
\end{enumerate}