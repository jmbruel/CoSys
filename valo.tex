\Mycomment{A titre indicatif : 2 pages pour ce chapitre.
Pr�senter les strat�gies de valorisation des r�sultats :
\begin{itemize}
\item   la communication scientifique,
\item   la communication aupr�s du grand public (un budget sp�cifique peut �tre pr�vu),
\item   la valorisation des r�sultats attendus,
\item   les retomb�es scientifiques, techniques, industrielles, �conomiques, \ldots
\item   la place du projet dans la strat�gie industrielle des entreprises partenaires du projet,
\item   autres retomb�es (normalisation, information des pouvoirs publics, ...),
\item   les �ch�ances et la nature des retomb�es technico- �conomiques attendues,
\item   l'incidence �ventuelle sur l'emploi, la cr�ation d'activit�s nouvelles, \ldots
\end{itemize}

Pr�senter les grandes lignes des modes de protection et d'exploitation des r�sultats
Pour les projets partenariaux organismes de recherche/entreprises, les partenaires devront conclure, sous l'�gide du coordinateur du projet, un accord de consortium dans un d�lai de un an si le projet est retenu pour financement.  
Pour les projets acad�miques, l'accord de consortium n'est pas obligatoire mais fortement conseill�.
}
